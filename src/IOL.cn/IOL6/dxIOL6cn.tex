\def \thisling{中文}
\def \thislang{中文}
\def \thistext{中文文本}
\def \olympiad{第六届国际理论, 数理及应用语言学奥林匹克竞赛}
\def \pgheader{第六届国际语言学奥林匹克竞赛}
\def \bulgaria{保加利亚}
\def \sunbeach{阳光海滩}
\def \olydates{2008年8月4 — 9日}
\def \probindl{个人赛题目}
\def \solsindl{个人赛解答}
\def \probteam{团体赛题目}
\def \soluteam{团体赛解答}
\def \probword #1{题 \##1}
\def \pontword{分}
\def \rulesmot{规则}
\def \regulats{解答规则}
\def \regulatx{毋需抄题. 将不同问题的解答分述于不同的答题纸上. 每张纸上注明题号、座位号和姓名}
\def \towarrant{否则答题纸可能被误放或遗失.}
\def \regulaty{解答需详细论证. 无解释之答案, 即便完全正确, 也会被处以低分.}
\def \editorsz{编者}
\def \edinchef{主编}
\def \goodluck{祝你好运}
\def \answersp{答案}
\def \fordinto #1{翻译成#1}
\def \outdrehu #1{翻译 #1}
\def \corrcorr{找出正确的对应关系}
\def \transcri{转写这些词}
\def \writorth{用 Listuguj 拼写法拼写}
\def \macschwa #1{In the {\tscr} #1 $\approx$ \word{o} in \word{abbot}}
\def \zoqwedge{$\approx$ \word{u} in \word{but}}
\def \oshiroko{is an open~\word{o}}
\def \eshiroko{$\approx$~\word{a} in \word{aspen}}
\def \onororth #1#2#3{#1 $\approx$ English \word{a} in \word{cat}, #2 = French \word{eu} or German \word{ö} (these letters stand for long vowels). #3 is read as a short #2}
\def \owumlaut{= French \word{eu} or German \word{ö}}
\def \chumlaut{= French \word{eu} and \word{u} (German \word{ö} and~\word{ü})}
\def \uyumlaut{= French \word{u} or German \word{ü}}
\def \nbsuperw{shows that the preceding consonant is pronounced with rounded lips}
\def \ruleforw #1#2{The letter #1 stands for a rounding of the lips after a consonant and for the sound #2 otherwise}
\def \rquvular #1#2{The letter #1 denotes a `Parisian' \word{r} (pronounced far back in the mouth), and #2 stands for a \word{k}-like sound made in the same place}
\def \mandcons #1#2#3#4{#1 and~#2 are hard consonants similar to English \word{sh} in \word{shut} and \word{ch} in \word{chuck}, #3 and~#4 are soft consonants similar to \word{sh} in \word {sheet} and \word{ch} in \word{cheat}}
\def \zoqconsz #1#2{#1 $\approx$ \word{nds} in \word{hands}, #2 = \word{sh}}
\def \cjmicmac #1#2{#1 = \word{ch} in \word{church}, #2 = \word{j} in \word{judge}}
\def \spunvoid{are specific unvoiced consonants}
\def \velarnas{= \word{ng} in \word{hang}}
\def \palatnas{$\approx$ \word{ni} in \word{onion}}
\def \jotisjot{nay}
\def \jotsound{= \word{y} in \word{yay!}}
\def \tsaffric{$\approx$ \word{ts} in \word{hats} (pronounced as a single consonant)}
\def \frivelar{= \word{ch} in Scottish \word{loch}}
\def \dittovoi{is the same sound but voiced}
\def \glotstop{is a specific consonant (the so-called glottal {\stp})}
\def \retrflex{$\approx$ \word{d} and \word{t} in \word{word} and \word{art}, uttered with the tip of the tongue turned back}
\def \thorneth #1#2#3{English #1 in #2 and #3 respectively}
\def \singsyll{are pronounced as a single syllable}
\def \hetteken{标记}
\def \aspirath{indicates that the preceding {\stp} consonant is aspirated (pronounced with a puff of air)}
\def \longmark{表示长元音}
\def \apostrof #1{The apostrophe indicates length if it follows a vowel, and is read as #1 if it follows a consonant}
\def \schwason #1{is pronounced, though not written, between any consonant and a following sonorant consonant #1}
\def \schwabis{is also pronounced before a consonant cluster at the beginning of a word}
\def \rulestop #1#2#3{#1 are pronounced as \voicon{}s (#2) at the beginning of a word or between vowels and as voiceless consonants (#3) at the end of a word or next to another consonant}
\def \incanada{在加拿大}
\def \northcan{在加拿大北部}
\def \inchiapa{在墨西哥南部的恰帕斯州}
\def \spokenca #1#2{#2, 约有#1人使用该语言}
\def \drehname{利富语}
\def \drehspok #1{在新喀里多尼亞东部的利富岛上, 超过#1人在使用利富语}
\def \indrehus{在利富语中}
\def \cemuname{Cemuhî}
\def \cemuspok #1{Cemuhî is spoken by approx.\ #1 people on the east coast of New Caledonia}
\def \inukname{Inuktitut}
\def \geninukt{Inuktitut (Canadian Inuit) belongs to the Eskimo-Aleut family of languages}
\def \zoquenom{Copainalá Zoque}
\def \genzoque #1{The #1 language is of the Mixe-Zoque linguistic family}
\def \bothaune{Both languages belong to the Austronesian family}
\def \gemicmac{米克马克语是一种阿尔冈昆语}
\def \inmicmac{下面是以被称为李斯图古吉的拼写法拼写地米克马克语的单词, 它们的语音\tscr及汉语翻译}
\def \indrecem{The following are words and compounds in two languages of New Caledonia}
\def \giveword #1{The following are words in #1}
\def \inscalds #1{下面是古诺尔斯语诗歌的四个片段, 编写于公元九百年前后. 它们均遵循韵律 #1 (字面意义 ‘宫廷韵律’)}
\def \insentin #1{The following are sentences in #1}
\def \redrecem #1#2{And here are several words translated from #1 into #2}
\def \dqstanza{Given is a stanza in which 13 words are omitted}
\def \theirtra{their English translations}
\def \chaotict{given out of order}
\def \cdaskone #1#2#3#4{What do you think the words #1 mean in #2, and #3 in #4}
\def \cdasktwo #1#2#3#4#5{In #1 #2 is `#3' and #4 is `#5'}
\def \allitera{头韵}
\def \videstat{See the statement of the problem}
\def \allitert #1{#1 的一个主要规则是头韵}
\def \alliterh{每个押韵对句 (行对) 的第一行包含两个单词, 首音相同, 且第二行的第一个单词亦是由此音开头}
\def \vocallit #1{元音被认为与另一个元音或 #1 合头韵}
\def \nunirule{但这不是唯一的规则}
\def \mdiffers{不止一份上述文本的手抄本传承下来}
\def \mvariats{有时候在文本相同的地方出现了不同的单词, 学者需判断哪种变体为原文}
\def \huchoose{不同的考虑可能影响结论. 有时候诗律规则有助于确定某些错误变体}
\def \nonlybut #1#2#3{比如, 在行 #1 中我们不仅仅发现 #2, 也发现了 #3}
\def \outother #1#2{可以根据诗的结构排除, 但 #1 和 #2 都可接受, 因此人们需要其他理由来选择单词.}
\def \drotqtri #1#2#3{在行 #1 中, #2 和 #3 均出现在手抄本中, 但 #3 并不满足诗的要求.}
\def \vadrules #1{Describe the rules which are observed in a distich of #1}
\def \dqfiller{The following list contains (in alphabetical order) all 13 omitted words and two words which do not belong in stanza V}
\def \fillgaps{Fill in the gaps in stanza V}
\def \wwonorse{古诺尔斯语是一种北日耳曼语,在公元七百年至一千一百年左右使用}
\def \onornorm{All samples of poetry in the problem are given in a normalized orthography and conform to the rules of the genre}
\def \qtysylla{Number of syllables. Each line contains 6 syllables}
\def \inrhymes{Internal rhyme}
\def \inrhymep{Let us denote the vowels (and diphthongs) in each line by}
\def \inrhymeq #1#2{At least one consonant immediately following V$_5$ must immediately follow #1 #2}
\def \inrhymer{Also, in even lines}
\def \egdrotts #1{For instance, cf.\ lines #1 (alliteration is marked in boldface, internal rhyme by underlining)}
\def \excesswd{Leftover words}
\def \autelexp{圣地是教堂的最重要, 神圣的部分}
\def \autelmot{圣地}
\def \abeilles{swarm of bees}
\def \abeillex{蜜蜂 (单数)}
\def \abeimult{蜜蜂 (复数)}
\def \animalwd{动物}
\def \aquagran{水}
\def \banamult{香蕉 (复数)}
\def \bananasz{bunch of bananas}
\def \bananawd{香蕉 (单数)}
\def \biblemot{Bible}
\def \boiremis{饮用 (动词)}
\def \bonemult{骨头 (复数)}
\def \boneword{骨头 (单数)}
\def \calendar{日历}
\def \carrelet{钻子}
\def \costemot{coast}
\def \dimanche{星期天}
\def \dormimot{睡觉 (动词)}
\def \ecclesia{教堂}
\def \esperonx{踢马刺}
\def \esperony{tool to poke animal}
\def \fourchet{叉子}
\def \frontier #1{#1 border}
\def \holyhome{holy}
\def \holyjour{holy}
\def \holypost{holy}
\def \holytome{holy}
\def \homegran{house}
\def \homeword{house}
\def \jourmult{days}
\def \jourword{day}
\def \kurkalem{pencil}
\def \litparol{bed}
\def \multitud #1{multitude of #1}
\def \murparol{wall}
\def \nuitgran{night}
\def \piquemis{to poke}
\def \piquemot{to poke}
\def \tool #1{tool #1}
\def \postdorm{place to sleep}
\def \postword{place}
\def \schrimis{to write}
\def \schrimot{to~write}
\def \skeleton{skeleton}
\def \tomeword{book}
\def \twilight{twilight}
\def \verremot{cup}
\def \shamanex{A shaman is a priest, sorcerer and healer in some cultures}
\def \thejager{这个猎人}
\def \deshaman{这个萨满}
\def \sinukoer{你的狗}
\def \inuktitq{你来过了}
\def \inuktitk{你摔倒了}
\def \inuktitl{你伤了你自己}
\def \shottest #1{你射了#1}
\def \inuktitr{你治好了一个猎人}
\def \inuktitc{你刺了那条狗}
\def \adogword{一条狗}
\def \tteacher{这个老师}
\def \inuktitj{治好了一个医生}
\def \inuktitw{伤了一个老师}
\def \inuktits{伤了你}
\def \inuktitb{看见过你}
\def \inuktitg{刺了那匹狼}
\def \inuktiti{射了一只北极熊}
\def \inuktitf{这个男孩射了那个医生}
\def \inuktite{这条狗伤了你的老师}
\def \inuktitm{这匹狼看见过你的萨满}
\def \inuktitu{你的狼摔倒了}
\def \inuktith{这只北极熊来过了}
\def \inuktitn{你的北极熊伤害了一个男孩}
\def \inuktita{这个医生治好了你}
\def \inuktitd{你的医生治好了你的男孩}
\def \inuktito{你的猎人治好了他自己}
\def \inuktitt{这位老师看到了那个男孩}
\def \withword{`with'}
\def \forparol{for}
\def \abovemot{上方}
\def \renderpl{plural}
\def \zadiente{for the tooth}
\def \zacourge{for the squash}
\def \courges{squashes}
\def \cucourge{with the squash}
\def \nakpat{a cactus}
\def \kittel{a kettle}
\def \zakittels{for the kettles}
\def \sukumguy{above the town}
\def \zakumguys{for the towns}
\def \balkan{a mountain}
\def \balkanes{mountains}
\def \cumokmok{with the corn}
\def \cumikittel{with my kettle}
\def \mimokmok{my corn}
\def \mipillar{my post}
\def \sumihimmel{above my sky}
\def \supillar{above the post}
\def \pillars{posts}
\def \tudientes{your teeth}
\def \cutunakpat{with your cactus}
\def \tukumguy{your town}
\def \tucaycay{your vine}
\def \himmel{a sky}
\def \poxkuy{a chair}
\def \zapoxkuy{for the chair}
\def \supoxkuys{above the chairs}
\def \zacaycay{for the vine}
\def \sucaycay{above the vine}
\def \shadow{a shadow}
\def \shadows{shadows}
\def \sushadows{above the shadows}
\def \cushadow{with the shadow}
\def \axeparol{斧头}
\def \unsafewd{不安全}
\def \arxangel{天使长}
\def \shoeword{钉蹄铁 \emph{(给马)}}
\def \spoonmot{匙}
\def \agentmot{印第安代理人}
\def \lawparol{法律}
\def \lieontop{躺在顶部}
\def \indianfe{印第安女人}
\def \oheadmot{上方, 在头顶上}
\def \stovemot{炉}
\def \foolishn{愚蠢 (名词)}
\def \springwa{泉水}
\def \raspberr{树莓}
\def \worships{崇拜}
\def \goatword{山羊}
\def \southmot{南方}
\def \snakemot{蛇}
\def \lookroon{环顾四周}
\def \therefor{所以, 因此}
\def \defobj #1{definite object}
\def \cedreord{The modifier follows its head in both languages}
\def \instruct{The Inuktitut sentences have the following general structure}
\def \wherinuk #1#2{where #1 are nouns and #2 is the verb}
\def \declinuk #1#2#3{If a noun gets the ending #1 when it is either a \defobj{} or a subject of a sentence that doesn't have a \defobj{}, it also gets #2 before the ending #3 when it is an in\defobj{}}
\def \sayyours #1#2#3#4{To say `your', #1 is replaced by #2, #3 by #4}
\def \zoquesfx{The noun suffixes seen in this problem are}
\def \suffinuk{The verb receives the following suffixes}
\def \jortinuk #1#2{#1 following a vowel or #2 following a consonant}
\def \endingin{an ending for the persons of the subject and the \defobj{}, if there is one}
\def \schemata{in the first two schemata}
\def \schematr{in the third schema}
\def \transref{一个没有宾语的及物动词当作反身动词处理}
\def \simnasal #1#2#3{After a nasal consonant (#1) the {\stp}s #2 become voiced (#3 respectively)}
\def \metathyk #1#2{If #1 comes after #2, the two sounds exchange places}
\def \possezoq #1#2{The possessive pronouns are #1 `my' and #2 `your'}
\def \prenasal{if the noun begins with a \stp, this consonant becomes voiced and the corresponding nasal appears before it}
\def \hh{送气}
\def \stp{塞音}
\def \frc{擦音}
\def \tscr {转写}
\def \transcrn{\tscr}
\def \fqts #1{反切\tscr#1}
\def \risetone{升}
\def \flattone{平}
\def \falltone{降}
\def \altotone{高}
\def \bajotone{低}
\def \Guangyun{\textsl{广韵}}
\def \xarakter{汉字}
%\def \chinname{Chinese}
\def \mandname{普通话}
\def \cantname{广东话}
\def \putongww{普通话是中国的官方语言, 基于北京方言. 约八亿五千万人使用该语言}
\def \wuandyue{九千万人使用吴语 (上海话), 七千万人使用广东话 (粤语).}
\def \whencomp #1{在编译#1时}
\def \homochin #1#2{在编译#1字典时 (#2), 汉语相当同质化}
\def \nophonet{由于汉字不是表音文字}
\def \oneastwo{该字典使用了一套简单的系统, 通过两个汉字来表示一个汉字的发音, 而读者理应知道前者的发音 (它们是常用字)}
\def \calledfq{这套系统叫做\emph{反切}.}
\def \hetechin{后来, 虽然汉语方言分化, 但许多古代的\fqts{}仍可使用, 只不过在不同的方言中有不同 (且更复杂) 的使用方法.}
\def \herecant{下面是一些反切{\tscr}. 每个汉字的广东话读音亦给出.}
\def \heremand{下面是另外一些{\tscr}, 但只给出了其普通话读音}
\def \canmando{下面是些汉字及其广东话和普通话的读音}
\def \canmanre{下面是另外一些汉字及其广东话和普通话的读音}
\def \hucanton{解释古代\fqts{}是如何应用于现代广东话的}
\def \huwasuse #1{#1\fqts{}是如何工作的}
\def \whatonly #1{在广东话中, 上述\tscr只有一个可以应用这条简单的老规则来得到正确的结果. 哪一个呢?}
\def \nownovoi #1{在多数当代汉语方言中 (包括广东话和普通话) 不存在\voicon{}, 除了响音 (#1)}
\def \ccvoiced #1{在#1编译时汉语存在其他\voicon{}, 它们后来并成对应的清音}
\def \fromvoid{浊}
\def \voicon #1{浊辅音}
\def \tounvoid{清}
\def \devofric #1#2{#1{\frc}变成#2{\frc}}
\def \devostop #1#2{#1{\stp}变成{\hh}或不{\hh}#2{\stp}}
\def \wuvoiced{浊音在吴语中得到了保留}
\def \wuvoicex #1#2#3#4{比如, 汉字 #1 在吴语中发作 #2, 在广东话中发作 #3, 在普通话中发作 #4}
\def \wherevoi #1{上一节中的哪些汉字#1首辅音为浊辅音}
\def \whyvdasp{这些\voicon{}在广东话中变得{\hh}或者不{\hh}取决于什么条件?}
\def \tredecar{在古汉语中存在四种声调, 但是只有三种出现在此题中}
\def \hutresex{解释这三种音调是如何演变出广东话的六种音调.}
\def \formanda{暂时忽略音调, 给出在普通话中使用古代\fqts{}的规则.}
\def \tonemiss{一些音调被移除}
\def \devemand{描述音调和浊首辅音是如何在普通话中演变的}
\def \humandar{可以总结出哪些在普通话中读\fqts{}的音调的规则?}
\def \rarewhat{一些初始的辅音和音调组合在现代普通话中极为罕见. 哪些呢?}
\def \supptone{判断出遗失的音调是哪些}
\def \readfoll #1{给出下面的\tscr#1的读法}
\def \incanton{在广东话中}
\def \inmandar{在普通话中}
\def \infohave{一些\tscr本不可读, 但这道题包含了足以读出它们的信息}
\def \triparts{汉语音节由三部分组成}
\def \onrimton{声母 (首辅音, 可能不存在如 3B), 韵母 (后面的所有音) 和声调}
\def \cantones #1#2{广东话音调可以认为存在两种不同的性质: 音高 (#1) 和轮廓 (#2).}
\def \socanton{若要在广东话中使用\fqts{}, A 的声母和声调音高将于 B 的韵母及声调轮廓组合}
\def \decanton #1#2#3{But if A’s (and X’s) tone is #1, X’s onset, if a \stp, must always be {\hh} if B’s (and X’s) tone is #2, and un{\hh} if it is #3}
\def \onsarhyb{Certainly the onset was from the A character, and the rhyme from B}
\def \aspirodd{But the aspiration rule is strange}
\def \noterken{Probably it was not part of the original fanqie system}
\def \tonfromb{Maybe the tone came from only one of the two characters? That has to be B, because the old rule should give correct results in only one {\tscr}.}
\def \urfanqie{Thus the original simple rule for fanqie was: A’s onset is combined with B’s rhyme and tone}
\def \onlyelve{Only {\tscr} 11 can be read now using this rule}
\def \sonorlow #1{Looking at the syllables with a sonorant onset, we see that they are always in a low tone (#1)}
\def \voitolow{Assuming that all \voicon{}s evolved alike in Cantonese, we may conclude that what is in a low tone now, had a voiced onset earlier}
\def \volosupp{This is also true of the character of the example from Wu. What is said in \textbf{(d)} supports this idea.}
\def \sherevoi{Thus the characters whose onsets were voiced are}
\def \ifiatall{if it had an onset at all}
\def \voidevel #1#2{Voiced {\stp}s became {\hh} if the tone was #1, and un{\hh} if it was #2}
\def \contoura{The contours of the Cantonese tones correspond to the three tones of Classical Chinese}
\def \nuheight{tone height is an innovation brought about by the evolution of the \voicon{}s.}
\def \explaict #1{因此我们可以解释为什么\fqts{}#1这样读}
\def \samehigh{The X character has the same tone height as A because it got its onset from A, and height in Cantonese is determined by the voicing of the onset in Classical Chinese}
\def \otherasp{But if the onset was a voiced \stp, it could evolve in different ways in X and A, because its aspiration was determined by the tone contour, which X got from B, and it could differ from A’s contour.}
\def \mandcomp{In Mandarin onsets and rhymes are not combined in such a straightforward way as in Cantonese}
\def \conoccur #1#2#3{It can be noted that after #1 we always find #2, whereas #3 are never followed by these vowels}
\def \dejaknow{We already know that the onset came from A and the rhyme from B}
\def \whenrest{When the constraint above came into being}
\def \ruharden #1#2#3#4{#1 was lost and #2 became #3 after #4}
\def \rusoften #1#2#3{#1 became #2 before #3}
\def \usethese #1{#1使用\fqts{}同样需要应用如上规则}
\def \onsetcis #1#2{if #1's onset is #2 and}
\def \rhymebei #1#2{B's rhyme starts with neither #1 nor #2}
\def \onsetane{A's onset is none of these}
\def \indeterx #1{we can't determine what X's #1 is}
\def \acconset{onset}
\def \accrhyme{rhyme}
\def \vocanton{On the basis of the tone of the Cantonese syllable we can determine whether the onset was voiced or not in Classical Chinese}
\def \mandtone{In Mandarin the tones developed as follows}
\def \mandtons #1#2#3{#2 if the onset was #1, #3 otherwise}
\def \voinoson{voiced but not a sonorant}
\def \unvoiced{voiceless}
\def \decontur{We see that the contour is not preserved here}
\def \somanton #1{In \fqts{s} read #1 the tones work as follows}
\def \legecons #1#2#3#4{Here #1 stands for a sonorant, #2 for a \frc, #3 for an un{\hh} and #4 for an {\hh} \stp}
\def \dimanton{Thus most of the time X's tone in Mandarin can't be derived unambiguously from A's and B's tones, though in some cases it can.}
\def \rarethat{Syllables with a sonorant onset and tone 5 or with an un{\hh} onset and tone 35 should not exist in Mandarin (if they do, then the rules must have had exceptions).}
\def \chintons{每个汉语方言都有固定数目的音调 (每个音节发音的旋律)}
\def \chaotons{本题使用了语言学家赵元任提出的系统, 其用数字 1 (最低) 到 5 (最高) 来标记音高的五级, 并将音调转写成连续的音级}
\def \havetons{你所需的全部音调都在本题中出现了}
\def \zidenote{如果你不想写汉字, 你可以使用{\tscr}的序号并指明具体的汉字来指代它们: X (被转写的)}
\def \nrintscr #1{#1用于{\tscr}的汉字}
\def \premiere{第一个}
\def \deuxieme{第二个}
\def \nnovowel #1{注意普通话的 #1 汉字的读法不包含元音}
\def \enquetea{你做了哪些题目?}
\def \enqueteb{你最喜欢哪道题?}
\def \enquetec{你觉得哪道题最难?}
\def \enqueted{你觉得哪道题最简单?}
\def \keredomo{However,}
\def \alwaysso{always}
\def \eg{如}
\def \et{和}
\def \au{或}
\def \Vhimself{\textsf{V} (himself)}
\def \aYrender{a \textsf{Y}}
\def \theYrend{the \textsf{Y}}
\def \solo{only}
\def \just{just}
\def \rite{right}
\def \como{like}
\def \cosi{as if}
\def \ABname{Alexander Berdichevsky}
\def \BBname{Bozhidar Bozhanov}
\def \SBname{Svetlana Burlak}
\def \DGname{Dmitry Gerasimov}
\def \XGname{Ksenia Gilyarova}
\def \IGname{Ivaylo Grozdev}
\def \SGname{Stanislav Gurevich}
\def \IDname{Ivan Derzhanski}
\def \AHname{Adam Hesterberg}
\def \BIname{Boris Iomdin}
\def \IIname{Ilya Itkin}
\def \RPname{Renate Pajusalu}
\def \APname{Alexander Piperski}
\def \MRname{Maria Rubinstein}
\def \LFname{Ludmilla Fedorova}
\def \TTname{Todor Tchervenkov}
\def \LMSname{刘闽晟}
\def \edinames{\ABname, \BBname, \SBname, \IDname\ (\edinchef), \LFname, \DGname, \XGname, \IGname, \SGname, \AHname, \BIname, \IIname, \RPname, \APname, \MRname, \TTname}
\def \whowroti{\LMSname}
\def \whowrotj{\LMSname}
