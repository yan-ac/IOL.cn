\def \thisling{English}
\def \thislang{English}
\def \thistext{English text}
\def \olympiad{Seventh International Olympiad in Theoretical, Mathematical and Applied Linguistics}
\def \pgheader{Seventh International Olympiad in Linguistics}
\def \xPoljska{Poland}
\def \yWroclaw{Wrocław}
\def \olydates{26–31 July 2009}
\def \probindl{Individual Contest Problems}
\def \solsindl{Individual Contest Solutions}
\def \probteam{Team Contest Problem}
\def \soluteam{Team Contest Solution}
\def \probword #1{Problem \##1}
\def \pontword{points}
\def \regulats{Rules for writing out the solutions}
\def \regulatx{Do not copy the statements of the problems. Write down your solution to each problem on a separate sheet or sheets. On each sheet indicate the number of the problem, the number of your seat and your surname}
\def \towarrant{Otherwise your work may be mislaid or misattributed.}
\def \regulaty{Your answers must be well-argumented. Even a perfectly correct answer will be given a low score unless accompanied by an explanation.}
\def \editorsz{Editors}
\def \edinchef{editor-in-chief}
\def \goodluck{Good luck}
\def \answersp{Answers}
\def \giphratr #1{The following are phrases in English and their translations into the #1 language}
\def \gisentra #1{The following are sentences in #1 and their English translations}
\def \fordinto #1{Translate into #1}
\def \fallways #1{\fordinto #1 in all possible ways}
\def \fillgaps{Fill in the gaps}
\def \spokenca #1#2{It is spoken by approx.\ #1 people #2}
\def \introman{In the Roman script}
\def \oshiroko #1{#1 $\approx$~\word{a} in \word{hall}}
\def \eshiroko #1{#1 $\approx$~\word{a} in \word{hat}}
\def \aconsons{are consonants}
\def \jotsound #1{#1~= \word{y} in \word{yay!}}
\def \djaffric #1{#1~= \word{j} in \word{judge}}
\def \chaffric #1{#1~= \word{ch} in \word{church}}
\def \velarnas #1{#1~= \word{ng} in \word{hang}}
\def \retrflex #1#2#3{#1,~#2 and~#3 $\approx$ \word{n}, \word{sh} and \word{t} in \word{barn}, \word{marsh} and \word{art}, uttered with the tip of the tongue turned back}
\def \syllabir #1{#1~is a vowel similar to the middle sound in American English \word{bird}}
\def \burmethe #1#2{#1~$\approx$ English~#2 in \word{with}}
\def \glotstop #1{#1~is a consonant (the so-called glottal stop)}
\def \longmark #1{The mark~#1 denotes vowel length}
\def \nasaleng #1{#1~indicates that the preceding vowel is nasal}
\def \aspirate #1{#1~indicates that the preceding consonant is aspirated (pronounced with a puff of air)}
\def \hilotone #1#2{The marks #1 and #2 indicate high and low tone (level of voice when pronouncing the syllable), respectively}
\def \nonemidt{if neither is present, the syllable has middle tone}
\def \burorder{Here are the names of 24 Burmese children and their dates of birth}
\def \sixmoreb #1#2{On #1, and #2 six other Burmese children were born}
\def \nomchaos{Here are their names}
\def \whowhenb{Who was born when}
\def \burmasol{We can see that the names of the children born on the same day of the week have similar first sounds}
\def \nosunday{there are no Sunday-born children in the data, nor any names beginning with vowels}
\def \ladsnoms{boys}
\def \lassnoms{girls}
\def \bairname{name}
\def \borndate{date of birth}
\def \oldindin{Given are Old Indic word stems which are thought to preserve the most ancient (Indo-European) position of the stress}
\def \oldindiv{They are divided into root and suffix by a hyphen}
\def \oldindas{Explain why it is not possible to use these data to determine the placement of the stress of the following stems}
\def \oldindan{Indicate the stress of the word stems given below}
\def \wostress #1{The stressed vowel bears the mark~#1}
\def \washaqua{washing water}
\def \gerontok{gerontocracy}
\def \hyenamot{hyena}
\def \hippopot{hippopotamus}
\def \slatemot{slate}
\def \bigsieve{wide-meshed sieve}
\def \bilakoro{uncircumcised boy}
\def \surprise{unexpectedness}
\def \epistola{letter, missive}
\def \causemot{because}
\def \hailword{hail}
\def \matchman{match-seller}
\def \niechmot{may it be that}
\def \arcoiris{rainbow}
\def \advertis{advertising}
\def \rustword{rust}
\def \lamplite{light (of a lamp)}
\def \xianpope{Christian priest}
\def \wolowolo{a kind of midges}
\def \wolomesi{honey from such midges}
\def \jamanakE{the joys, pleasures of youth}
\def \wolfword{wolf}
\def \enemymot{foe}
\def \smokemot{smoke}
\def \stepword{step}
\def \premiere{first}
\def \ghrishvi{exuberant}
\def \sleepmot{sleep}
\def \cloudmot{cloud}
\def \nakedmot{naked}
\def \foamword{foam}
\def \lordword{lord}
\def \messengr{messenger}
\def \hymnword{hymn}
\def \castrate{castrated}
\def \murderwd{killing}
\def \drauxtwd{drinking, draught}
\def \heatword{heat}
\def \desirewd{desire}
\def \partword{share}
\def \soilword{earth, soil}
\def \movement{walk}
\def \cordword{cord}
\def \burdenwd{burden}
\def \praisewd{praise}
\def \effusion{effusion}
\def \activewd{mobile}
\def \deadweap{deadly weapon}
\def \plouxing{tillage}
\def \rollword{rolling}
\def \aquaskin{leather bag}
\def \prosperi{prosperity}
\def \demolixy{demolisher}
\def \horsebak{sitting on horseback}
\def \mornlixt{morning light}
\def \pterocle{sandgrouse}
\def \kindbird{a kind of bird}
\def \euphemis{euphemism}
\def \Monday{Monday}
\def \Tueday{Tuesday}
\def \Wedday{Wednesday}
\def \Thuday{Thursday}
\def \Friday{Friday}
\def \Satday{Saturday}
\def \Sunday{Sunday}
\def \inmanden{The following are words of the Maninka and Bamana languages written in the N'Ko and the Roman script and their English translations}
\def \kantenko{The N'Ko script was invented in 1949 by the Guinean enlightener Soulemayne Kante}
\def \dirnkorl{The N'Ko script is written and read from right to left}
\def \alphanko{The script is an alphabet: each letter stands for a consonant or a vowel}
\def \linklett{The letters within a word are joined}
\def \tontilde{A tilde above a vowel letter means low tone, its absence means high tone}
\def \tonemidd{But a vowel has middle tone if it is marked in the same way as the one before it (if both either have or lack tildes)}
\def \nkoshort{If two adjacent syllables have the same vowel and both letters should have a tilde or neither should have one according to the rules, only the second vowel is written}
\def \gmanding{The Bamana and Maninka languages belong to the Manding group of the Mande language family}
\def \maliguin{They are spoken in Mali, Guinea and other countries in West Africa}
\def \proxling{These languages are very close to one another; the distinction between them is of no consequence to the problem}
\def \burmtran{The Burmese names are given in a simplified Roman transcription}
\def \nahuorth{The Nahuatl sentences are given in a simplified orthography}
\def \verberst{The Nahuatl sentences begin with the predicate}
\def \nounexts #1{The subject and object (or objects) follow in any order, preceded by #1 (a definite article)}
\def \nahupref{The verb receives the following prefixes}
\def \nahusuff{As well as the following suffixes}
\def \lengprec #1{with lengthening of a preceding #1}
\def \chanprec #1#2{with change of a preceding #1 to #2}
\def \Sb{subject}
\def \Ob{object}
\def \bisOb{another object}
\def \Vintrans{intansitive verb}
\def \Vtransit{transitive verb}
\def \suppletV{Often the same action with and without an object is expressed by different verbs}
\def \reNahua{Classical Nahuatl was the language of the Aztec Empire in Mexico}
\def \lgNahua{Nahuatl}
\def \lgSulka{Sulka}
\def \geSulka{Sulka belongs to the East Papuan language family}
\def \eastnewb{in East New Britain Province in Papua New Guinea}
\def \loSulnum{Here are the words from which the Sulka language constructs its numerals}
\def \meansadd{addition}
\def \meansdup{doubling}
\def \singular{singular}
\def \pluralup{plural (from 3 on)}
\def \noundiff{Nouns have different forms for the two numbers}
\def \specnums{There are separate words for a foursome of coconuts, for a twosome and foursome of breadfruit}
\def \cocSolve #1{coconut\ifnum1<#1s\fi}
\def \betSolve #1{betel nut\ifnum1<#1s\fi}
\def \panSolve #1{breadfruit\ifnum1<#1s\fi}
\def \yamSolve #1{yam\ifnum1<#1s\fi}
\def \notebetl{Betel nuts are actually seeds of a certain kind of palm}
\def \noteyams{Yam is the edible tuber of the tropical plant of the same name}
\def \rematole{Atole is a cornstarch-based hot drink}
\def \thexatol{the atole}
\def \thexwine{the wine}
\def \thexbook{the book}
\def \medicins{the healer}
\def \medicinb{for the healer}
\def \merchans{the merchant}
\def \merchant{the merchant}
\def \carpters{the carpenter}
\def \carpterx{the carpenter}
\def \carptert{the carpenter}
\def \kobietat{the woman}
\def \kobietax{the woman}
\def \kobiecie{to the woman}
\def \canzonex{the song}
\def \schlaeft{sleeps}
\def \prepares #1{prepares #1}
\def \tumakunw #1{you prepare #1 for somebody}
\def \nahuatlc{beats you for somebody}
\def \nahuatld{beats somebody for you}
\def \tikwitek{you beat him}
\def \ichdrink #1{I drink #1}
\def \hedrinks{drinks}
\def \umeleave #1{you leave #1 for me}
\def \tikmakam #1{you give #1 to him}
\def \titzahtz{you shout}
\def \tucantas{you sing}
\def \egocanto #1#2{I sing #1#2}
\def \singsong #1{sings #1}
\def \iloveyou{I love you}
\def \nisprech #1#2{I say #1#2}
\def \yoquiero #1{I want #1}
\def \nihuetzi{I fall}
\def \applicat{do for …}
\def \causatif{make …}
\def \caudrink #1{makes #1 drink}
\def \nicaunek #1#2{I make #1 want #2}
\def \ticausat #1#2{you make #1 #2}
\def \zulieben #1{love #1}
\def \zulassen #1{leave #1}
\def \wifellun #1{the woman makes #1 fall}
\def \causeste #1{makes you #1}
\def \tetolass #1{leave #1}
\def \sleepste{sleep}
\def \causesme #1{makes me #1}
\def \metomake #1{prepare #1}
\def \shoutsme{shout}
\def \hij{he}
\def \ninech{1st person sg}
\def \timitz{2nd person sg}
\def \cuq{3rd person sg}
\def \tet{somebody}
\def \te{\tet}
\def \tla{something}
\def \wennword{If the stop consonant in the root}
\def \stpinrdx #1{is #1}
\def \heliline{voiced}
\def \unvoiced{voiceless}
\def \vwlinsfx #1{and the vowel in the suffix is #1}
\def \stressed #1{the stress is on #1}
\def \radixloc{the root}
\def \sufixloc{the suffix}
\def \astprule{This rule holds if the root contains precisely one stop consonant}
\def \twonilup #1#2{If there are two (#1), or if there are none (#2), the place of the stress can't be determined}
\def \ABname{Alexander Berdichevsky}
\def \BBname{Bozhidar Bozhanov}
\def \SBname{Svetlana Burlak}
\def \DGname{Dmitry Gerasimov}
\def \XGname{Ksenia Gilyarova}
\def \IGname{Ivaylo Grozdev}
\def \SGname{Stanislav Gurevich}
\def \IDname{Ivan Derzhanski}
\def \AHname{Adam Hesterberg}
\def \BIname{Boris Iomdin}
\def \RPname{Renate Pajusalu}
\def \APname{Alexander Piperski}
\def \MRname{Maria Rubinstein}
\def \LFname{Ludmilla Fedorova}
\def \TTname{Todor Tchervenkov}
\def \MCname{Maria Cydzik}
\def \EKname{Evgenia Korovina}
\def \edinames{\ABname, \BBname, \IDname, \LFname, \DGname, \XGname, \SGname, \AHname, \RPname, \APname, \TTname\ (\edinchef)}
\def \whowroti{\BBname, \IDname, \AHname, \APname, \TTname}
\def \whowrotj{\BIname}
\def \quoted #1{“#1”}
\def \enqueten{Name}
\def \enquetep{Place}
\def \enquetea{What problems did you work on?}
\def \enqueteb{What problem did you like best?}
\def \enquetec{What problem did you find hardest?}
\def \enqueted{What problem did you find easiest?}
\def \genViet{Vietnamese belongs to the Austro-Asiatic language family}
\def \specamln #1{It is spoken by some #1 mln people in Vietnam}
\def \vietones{Vietnamese has six tones (melodies in one of which every syllable is pronounced)}
\def \arvowels{are vowels}
\def \palatnas{$\approx$ \word{gn} in \word{cognac}}
\def \retrfley #1#2#3{#1,~#2 and~#3 $\approx$ \word{r}, \word{sh} and \word{t} in \word{rye}, \word{marsh} and \word{art}, uttered with the tip of the tongue turned back}
\def \vietnach{$\approx$ \word{ch} in \word{cheat}}
\def \vietnadj{$\approx$ \word{j} in \word{jeep}}
\def \glottalh{= English~\word{h}}
\def \frivelar{= \word{ch} in Scottish \word{loch}}
\def \dittovoi{is the same sound but voiced}
\def \aspirath{is an aspirated~\word{t}}
\def \vietcons #1#2#3#4#5{#1 = \word{k}, #2 = \word{d}, #3 = \word{z}, #4 = \word{f}, #5 = \word{s} (voiceless)}
\def \marktone #1#2{One tone is not marked at all, the other five tones are marked by a diacritic above (#1) or below (#2) the vowel}
\def \listfreq{Here is a list of the 50 most frequent words of the Vietnamese language with their occurrences in a corpus (text collection) of one million words}
\def \tranmost{Translate as much as you can from the first ten reading units of a Vietnamese course for advanced beginners given below}
\def \allyfive{You will find all of the above words except five in the reading units}
\def \hixlited{These words are highlighted in the texts}
\def \alphword{Here are the words from the list that occur in the texts, in alphabetical order}
\def \seelemap{see the map}
\def \texttita{My Room}
\def \texttitb{Mr Nam Studies Korean at Hanoi University}
\def \texttitc{Mr Lee Comes to Vietnam}
\def \texttitd{Van Hung Works for \emph{Offo} Company}
\def \texttite{My Family}
\def \texttitf{I Live in Ho Chi Minh City}
\def \texttitg{Restaurant}
\def \texttith{Souvenir Shop in Hue City}
\def \texttiti{Tickets to Vietnam}
\def \texttitj{Hotel \emph{Sao Mai}}
