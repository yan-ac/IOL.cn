\def \thisling{English}
\def \thislang{English}
\def \thistext{English text}
\def \olympiad{Ninth International Olympiad in Linguistics}
\def \xcountry{United States of America}
\def \yvillage{Pittsburgh}
\def \Julyname{July}
\def \olydates #1#2#3#4{#1–#2 #3 #4}
\def \probindl{Individual Contest Problems}
\def \solsindl{Individual Contest Solutions}
\def \probteam{Team Contest Problem}
\def \soluteam{Team Contest Solution}
\def \Teamword{Team}
\def \probword #1{Problem \##1}
\def \pontword{points}
\def \regulats{Rules for writing out the solutions}
\def \regulado{Do not copy the statements of the problems}
\def \regulare{Write down your solution to each problem on a separate sheet or sheets}
\def \regulami{On each sheet indicate the number of the problem, the number of your seat and your surname}
\def \towarrant{Otherwise your work may be mislaid or misattributed}
\def \regulaty{Your answers must be well-argumented}
\def \regulatz{Even a perfectly correct answer will be given a low score unless accompanied by an explanation}
\def \editorsz{Editors}
\def \edinchef{editor-in-chief}
\def \goodluck{Good luck}
\def \rulesmot{Rules}
\def \answersp{Answers}
\def \andtrans{as well as their English translations}
\def \chaotict{in arbitrary order}
\def \fordinsg #1{Translate into #1}
\def \corrcorr{Determine the correct correspondences}
\def \filmties{Fill in the vacant cells}
\def \transall{If in some cases you believe that more than one translation is possible, give them all}
\def \isavowel{is a vowel}
\def \arvowels{are vowels}
\def \atwovocs{\arvowels}
\def \isaconst{is a consonant}
\def \aconsons{are consonants}
\def \atwocons{\aconsons}
\def \aeligatu #1{#1 $\approx$~\word{a} in \word{crack}}
\def \eshiroko #1{#1 $\approx$~\word{e} in \word{bed}}
\def \cyqueska #1{#1~= \word{k}}
\def \jotsound #1{#1~= \word{y} in \word{yum}}
\def \wawsound #1{#1~= \word{w} in \word{win}}
\def \chaffric #1{#1~= \word{ch} in \word{church}}
\def \glotstop #1{#1 is a consonant (the so-called glottal stop)}
\def \tonmarks #1{The marks~#1 denote tones}
\def \longmark #1{The mark~#1 denotes vowel length}
\def \camacron{called a macron (pl.\ macra)}
\def \belongsto #1#2{#1 belongs to the #2}
\def \toCMande{Central group of the Mande language family}
\def \toNGerma{Northern subgroup of the Germanic languages}
\def \spokenca #1#2{It is spoken by approx.\ #1 people #2}
\def \inLbrSle{in Liberia and Sierra Leone}
\def \iFaroetc{in the Faroe Islands and elsewhere}
\def \thelgMez{Menominee}
\def \introMez{Given are verb forms of the Menominee language}
\def \mezvestr{The Menominee verb forms have the following structure}
\def \rewheMez{The Menominee Indians live in Wisconsin, USA}
\def \retheMez #1{They number #1 people}
\def \elderMez{but the eponymous language of the Algonquian family is only spoken by a few dozen of the oldest among them}
\def \altheMez{although effort has been put lately into expanding its teaching and use}
\def \ThelgVai{Vai}
\def \thelgVai{Vai}
\def \introVai{Given are phrases in the Vai language}
\def \existerr #1{There is an error in the Vai phrase #1}
\def \corrtran #1{Correct it and translate the phrase into #1}
\def \erstsyll{In the first syllable}
\def \bnvowels{Between vowels}
\def \inN{in a noun}
\def \inV{in a verb}
\def \applerst{The first applicable rule is applied}
\def \ThelgFar{Faroese}
\def \inFaroes{The following are words of the Faroese language written in the regular orthography and in phonetic transcription}
\def \fillgaps{Fill in the gaps}
\def \descruls{Describe the rules you used}
\def \indetscr{In the transcription}
\def \theNahua{Nahuatl}
\def \inNahua{Given are words in Nahuatl}
\def \reNahua{Classical Nahuatl was the language of the Aztec Empire in Mexico}
\def \kenaq{we and you}
\def \rootwAEp{begin}
\def \kewAEpeqtaq{begin}
\def \wAEpAhpew{begins laughing}
\def \kewAEpAnaehkaeq{begin to dig}
\def \wAEpohnaew{begins walking}
\def \newAEpahtan{begin to eat it}
\def \rootkaw{down}
\def \nekAwAhpem{fall over laughing}
\def \nekAweqtam{lie down}
\def \kekAwaeq #1{lay it flat #1}
\def \kawam #1{fells it #1}
\def \rootkEsk{through}
\def \kekEskaeq #1{chop it through, break it through #1}
\def \kEskam #1{breaks it through #1}
\def \kekEskahtaeq{bite it through}
\def \rootket{out}
\def \nekAEtan #1{pry it out #1}
\def \ketam #1{takes it out #1}
\def \kekAEtohnaeq{walk out}
\def \ketOhnaew{walks out}
\def \rootpahk{off}
\def \pahkAEsam{cuts it off}
\def \nepAhkaln #1{break it off, tear it off #1}
\def \rootpAhk{open}
\def \pAhkeqtaw{opens up}
\def \pAhkam #1{opens it #1}
\def \nepAhkarn #1{open, uncover it #1}
\def \rootpIt{hither}
\def \nepItohnaem{walk here (to this place)}
\def \kepItahtaed{come eating it}
\def \kepItahtaeg{bring it in our mouths}
\def \pItam{passes it here}
\def \roottaw{pierce}
\def \tawAnaehkaew{digs a hole}
\def \tawAEsam{cuts a hole in it}
\def \ketAwahtaeq{bite, gnaw a hole in it}
\def \netAwan #1{pierce it #1}
\def \rootwack{around}
\def \newAckesan{cut around it}
\def \wackOhnaew{walks roundabout, by a detour}
\def \mezen{by hand}
\def \mezah{by tool}
\def \suffaht{by mouth}
\def \suffes{cutting}
\def \suffAhpe{laughing}
\def \sAnaehkae{digging}
\def \suffohnae{walking}
\def \pAhkaheg{by raising a lid or opening a door}
\def \Vintrans{intransitive verb}
\def \Vtransit{transitive verb}
\def \iambegin{If both first vowels in the word are short, the second becomes long}
\def \eyzur{wealth}
\def \logi{flame}
\def \lwgur{liquid}
\def \skOgur{forest}
\def \swga{story}
\def \toygur{swallow, gulp, draught}
\def \vizur{wood, timber}
\def \bwga{hen bird}
\def \glaza{whirlwind}
\def \glwzur{embers}
\def \hugi{mind}
\def \koyla{cleft}
\def \lega{bed}
\def \mOza{froth or scum in pot with meat or fish}
\def \plAga{nuisance, affliction}
\def \skazi{damage, loss}
\def \vAgur{gulf}
\def \tegi{keep silent}
\def \trUgi{may (he) endanger}
\def \mi{I}
\def \deyzi{kill}
\def \rAzi{advise}
\def \ta{he}
\def \knozar{kneads}
\def \tregar{injures}
\def \vegur{raises}
\def \nad{they}
\def \gleza{make glad}
\def \kvwza{sing}
\def \mugu{must}
\def \rwza{speak}
\def \rUma{contain}
\def \spreiza{spread}
\def \viga{weigh}
\def \wga{frighten}
\def \me{we}
\def \nyimii{snake}
\def \ja{eye}
\def \kafa {shoulder}
\def \lEndE{vessel }
\def \gbomus #1{the fish’s #1}
\def \kOanjas #1{the eagle’s #1}
\def \kOanjalOOkEnji {the small eagle’s claw}
\def \nyimiileNlOO{the small baby-snake}
\def \kandOlEndElOO{the small airplane}
\def \leNlOOs #1{the small child’s #1}
\def \leNkundus #1{the short child’s #1}
\def \nyimiikundus #1{the short snake’s #1}
\def \gbomulEndEkundu{the short boat}
\def \kais #1{the man’s #1}
\def \kOanjaleNfa{the baby-eagle’s father }
\def \musugbomu{the woman’s fish}
\def \nyimiijaNgbomulEndE{the long snake’s boat}
\def \musujaNlOOkai{the tall woman’s brother }
\def \kaijaNlOOmusu{the tall man’s sister}
\def \kandOjaN {the high sky}
\def \worderNA{Adjectives follow their nouns}
\def \litleunl #1#2#3{A noun gets the ending #1 unless it has one of the suffixes #2, #3}
\def \Ngetmark #1#2{A noun (#1) gets the marker #2}
\def \orAhavan{or the adjective if there is one}
\def \unlinali{unless it is inalienably possessed}
\def \bodypart{body part}
\def \kinsterm{kinship term}
\def \thenposs{in the latter case it is preceded by the possessor}
\def \possessd{possessed}
\def \possessr{possessor}
\def \alienpos #1{Alienable possession is expressed by #1 between the possessor and the possessed}
\def \incompoN{In compound nouns}
\def \modihead{the left-hand part modifies the right-hand one}
\def \lastonlo{the last syllable has low tone}
\def \tzin #1{revered #1}
\def \acalf{water house}
\def \acalli{canoe}
\def \achildef #1{Water pepper (#1) is a wild plant}
\def \achilli{water pepper}
\def \atl{water}
\def \cacahuatl{cocoa}
\def \cacahuaatl{cocoa drink}
\def \cacahuatetl{cocoa bean}
\def \callah{village}
\def \calli{house}
\def \chilatl{chili water}
\def \chiladef{Chili water is a Aztec drink containing chili pepper}
\def \chilli{chili}
\def \collib{grandfather}
\def \collid{ancestor}
\def \conehuah{mother}
\def \conehuahcapil{little mother}
\def \conetl{child}
\def \oquichconeg{male child}
\def \oquichconetl{boy}
\def \oquichhuah{wife}
\def \oquichtl{man/husband}
\def \oquichtotolin{turkey-cock}
\def \tetl{stone}
\def \tetlah{stony ground}
\def \tehuaq{person who lives in a stony place}
\def \totoltetl{turkey egg}
\def \calhuah{master of house}
\def \acalhuah{canoe owner}
\def \cacahuahuah{possessor of cocoa}
\def \tehuag{possessor of stones}
\def \ahuah{possessor of water}
\def \conehuaf{one who has child(ren)}
\def \oquichhuaf{one who has a husband}
\def \huah #1{one who has #1}
\def \tlah #1{place of many #1}
\def \dimin{dimin}
\def \afteroth #1#2#3{#1 after #2, otherwise #3}
\def \LostSymb{The Lost Symbol}
\def \FIland{Finland}
\def \FRland{France}
\def \DEland{Germany}
\def \IEland{Ireland}
\def \NOland{Norway}
\def \ESland{Spain}
\def \SEland{Sweden}
\def \books{books}
\def \ISBNbook{ISBN book}
\def \whence{from where}
\def \cost{cost}
\def \euros #1{#1 euros}
\def \centa #1{#1 cents}
\def \cents #1{#1 cents}
\def \checksum{checksum}
\def \lastxsum{The last digit is always a checksum}
\def \gifulcod{Write out the full code}
\def \fullcode{full code}
\def \prodcode{product code}
\def \puzmagaz{puzzle magazine}
\def \bumpaper{toilet paper}
\def \smokelox{smoked salmon}
\def \farsteak{pork steak}
\def \sirsteak{sirloin steak}
\def \chollows{cholesterol-lowering spread}
\def \mopshead{mop head}
\def \storpack{packed in the store}
\def \storefns{in-store functions}
\def \inbarcod #1{The barcode language #1 is used in almost every country in the world}
\def \anospeak{yet nobody speaks it}
\def \subcod{subcode}
\def \subcodes #1{It has #1 main dialects or subcodes}
\def \nocodnil #1{but this problem is not concerned with subcode zero, which is effectively the same as the older language #1}
\def \isbarcod #1#2{#1 is barcode #2}
\def \nobarcod{This is not a barcode}
\def \sibarcod{This is a barcode}
\def \issubcod #1{it belongs to subcode #1}
\def \unsubcod #1{it belongs to a possible subcode of #1 which is not in use}
\def \UKbarcod #1{This barcode is from a packet of biscuits from the UK, and the number starts with the country code or system number for the UK, which is #1}
\def \rebarcod #1#2{Usually the first part of the code (#1) identifies the producer and the next part (#2) is chosen by the producer and identifies the product}
\def \gridrite{On the right the machine-readable part of the code has been enlarged and transferred onto a grid for ease of observation}
\def \morenums{Here are some more system numbers}
\def \barcodAf #1{Here are some facts about barcodes #1, in no particular order}
\def \barcodAz{Give the letter of the barcode in each case, and answer any other questions}
\def \barcodBa #1{Draw the (imaginary) barcode #1 in the grid that you will find on one of your sheets}
\def \barcodBe{Some of it has been filled in to help you}
\def \dagblaNO #1{The barcode below is from #1, a newspaper from Norway}
\def \hvaderNO{What is the system number or country code for Norway}
\def \barframe #1#2{The patterns of bars of unit width #1 (at both ends) and #2 (in the middle) frame two blocks of six digits}
\def \digitcod #1#2{Each digit is shown as four bars of widths #1, with a total width of #2}
\def \threecod #1#2{There are three codes for each digit, one of which (#1) is used on the right and two (#2) on the left}
\def \subcodep #1#2{The pattern of #1s and #2s on the left gives the subcode}
\def \allstart #1{Each pattern starts with #1}
\def \conthree #1{contains exactly three #1s}
\def \rightway{this indicates that the barcode is the right way up}
\def \elsemirr #1#2{otherwise it would start with #1, the mirror image of #2}
\def \feallbar #1{The problem features all possible patterns except #1}
\def \onpriceA{Only barcodes for meat, cheese, etc., which have random weights have the price included as part of the barcode}
\def \onpriceB{for the rest, the price is looked up from the store’s computer system}
\def \onpriceC #1{These are produced in-store (#1) and so do not have a standard layout}
\def \onpriceD{but in the two that are given in the problem the last four digits before the checksum are the price}
\def \upsidown{This barcode is upside down}
\def \startwiB #1#2{it starts with a #1, not with an #2}
\def \mustturn{so it must be turned over and written backwards}
\def \teamiont{Team Contest Instructions}
\def \teamiona #1{You will have 3 hours for the team problem, which you will be given in an envelope marked with your team name and #1}
\def \teamionb #1{Thirty minutes after starting the contest, you must submit your progress (#1)}
\def \ansinsix{answers or insights into the problem matter}
\def \nodyssey{but not the odyssey by which you found them}
\def \iconvert{in the envelope the problem came in}
\def \teamionc #1{You will then be given an envelope marked with #1 containing hints, in the form of answers to some of the questions}
\def \teamionz{insights into the problem matter, and problem-solving tips}
\def \Insights{Insights into the problem matter}
\def \teamiond #1#2{The same applies #1, and #2 minutes after starting the contest}
\def \barfinal #1{except that #1 minutes after starting the contest, you'll be given not hints but the complete solution}
\def \teamione{You will be given more credit for correct answers and insights submitted at earlier half-hours}
\def \teamionf #1{approximately #1 fewer points will be available every half hour}
\def \teamiong{You need not submit the same thing more than once}
\def \latignor{later identical submissions will be ignored}
\def \indepont{Submissions will be graded independently}
\def \teamionh{with points awarded for correct answers and deducted for incorrect ones}
\def \teampteg{For instance, if you submit a wrong answer at 30 minutes and correct it at 90 minutes}
\def \teamioni{you will lose points for the 30-minute submission and separately gain points for the 90-minute one}
\def \teamionj{The first answer sheet will vaguely describe the insights that will be given as hints}
\def \teamionk{These are not necessarily the only insights worth any credit}
\def \teamionl{You will not be allowed to look at earlier envelopes after they're submitted}
\def \makenots{so be sure to make notes for yourselves of anything you submit but might need later}
\def \teamionm{All the answers and a complete description of the problem matter fit on one page}
\def \teamionn{so writing should not be a burden}
\def \disminun{Distribute at 0 minutes}
\def \disminut #1{Distribute at #1 minutes}
\def \colminut #1{Collect at #1 minutes}
\def \ansminut #1{Answers as of #1 minutes}
\def \justhint{This is just a hint, not the sort of insight you'd get points for}
\def \follhint{The following insights will at some point be given as hints}
\def \signifof #1{the meaning of #1}
\def \soonhint #1{the constraints of the poetry, and (in 3 hints) the use of #1}
\def \introSkr{The following ten lines are incorrect examples of Sanskrit poetry}
\def \werewell{They were originally written correctly}
\def \mumacrod{but there have been five deleted macra}
\def \vomacrob{four added macra}
\def \deschang{three changed letters and two deleted words}
\def \onewhole{so that only one line remained unchanged}
\def \sylstand{No syllables have been added or lost}
\def \sinodeld{except in the deleted words}
\def \Addmacra{Words with added macra}
\def \Delmacra{Words with deleted macra}
\def \Delwords{Deleted words}
\def \dLetters{Changed letters}
\def \redelmot{Restore the two deleted words}
\def \remodlet{revert the three changed letters}
\def \redothem{remove the four added macra, and restore the five deleted macra}
\def \canrecon #1#2{We can restore the deleted macron over the #1 in #2 for metrical reasons}
\def \nonrecon #1#2{but to purge the added macron over the second #1 in #2 requires knowing the word}
\def \orcompar #1{or comparison with line #1}
\def \fortpoet{Fortunately, the poets wrote in such a way that it's possible to correct all the changes except that one without any prior knowledge of Sanskrit}
\def \trawrong{The translations correspond to the lines after the two words were deleted and the three letters changed, but before any macra were added or deleted}
\def \yanother{There is one more rule of transliteration relevant to the {\mtr} that you will have to discover}
\def \whataffd{What is the additional rule of transliteration}
\def \thingist #1{The additional rule of transliteration is that #1 are long vowels}
\def \eolength #1{The vowels #1 are long}
\def \barnomac{although they are written without macra}
\def \offmacra #1{The macra have been removed from the Sanskrit mnemonic #1}
\def \mnemodef{A mnemonic is a word or sentence that helps remembering something}
\def \pimnemon{How I wish I could recollect}
\def \whatsyll #1{Which syllables were #1}
\def \whatsyli #1#2{Which syllables in #2 were #1}
\def \isquaple #1#2{#1 consists of four #2s}
\def \quadusef #1{That line #1 is four of something is useful information even if you don't know what it's four of}
\def \whethere #1#2{Where there is #1, there is #2}
\def \ifatends #1#2#3#4{In the case where a #1 is at the end of #2 #3s, it is said to be #4}
\def \vadethet #1{is called #1}
\def \twolines #1#2{Two lines of #1 they call #2}
\def \ifthrees #1#2#3#4{If perchance there are a pair of #1s, a #2, and a #3, then it is #4}
\def \thinerst #1#2#3{#1 is that #2 in which the first is #3}
\def \bhujpray{The movement of the snake}
\def \gajagati{The gait of an elephant}
\def \pramANik{The little measure}
\def \pramANix{the little measure}
\def \Indravaz{Indra's thunderbolt}
\def \Indravaj{Indra's thunderbolt}
\def \Upendvaj{Upendra's thunderbolt}
\def \Upendrau{Indra and his younger brother Upendra are Hindu gods}
\def \CaCikalA{the ascending period of the moon}
\def \vidymAlA{the garland of lightning}
\def \madhumat{(that which is) full of honey}
\def \pancAmar{the fan made of five yak tails}
\def \hintcomp #1#2{We suggest that you compare lines #1, and lines #2}
\def \hintread{We suggest that everyone on the team read every part of the problem and all the hints, since some of them may be relevant in unexpected places}
\def \kendinna{If a scribe knew Sanskrit but not the mnemonic system}
\def \resthint #1#2{they're more likely to have changed, say, a #1 to a #2 than a basic Sanskrit word}
\def \poetinte{The names of some of the {\mtr}s correspond to the {\mtr}s in poetically interesting ways}
\def \eginluna #1{For instance, there are #1 days in the lunar cycle, starting with the new moon, when the moon waxes, followed by the full moon itself}
\def \nonguess{unless you manage to guess a {\mtr} from its name alone}
\def \writback{use the back of the paper if necessary}
\def \selfdesc{Each line describes the {\mtr} in which it's written}
\def \divindiv{Division into syllables ignores word divisions}
\def \sylladiv #1#2#3#4{A sequence of type #1 is divided as #2; of type #3, as #4}
\def \defSguru #1{A syllable is #1 if and only if it has a long vowel or a diphthong or ends in a consonant}
\def \descmetr #1{Each {\mtr} can be described by a \emph{unique} sequence of the consonants of #1}
\def \adelword{One of the deleted words is}
\def \thatlett #1{The #1}
\def \addedlen{had a macron added}
\def \deledlen{had its macron deleted}
\def \inWord #1{in the word #1}
\def \Lineword{Line}
\def \inLine #1{in line #1}
\def \Twosylls{The syllables}
\def \Syllsare #1#2#3{Syllables #1, and #2 are #3}
\def \inSyll #1{in syllable #1}
\def \teislett #1{The second #1}
\def \lastlett #1{The last #1}
\def \isbroken{is incorrect}
\def \wasfirst #1{was originally #1}
\def \sentowas #1#2{For instance, sentence #1 was originally #2}
\def \mnemexpl #1#2{Each other syllable of #1 represents a \emph{unique} sequence of \emph{three} #2 syllables}
\def \mnemexpn #1#2#3{Each of the first #1 syllables of #2 stands for the pattern of #3 syllables in that and the next two syllables}
\def \macresto{after restoring its macra}
\def \makemnem{To make a \mtr's mnemonic}
\def \tritridu #1{group syllables by threes, and mark the at most two extras at the end with #1}
\def \standfor #1{stand for #1}
\def \infacteo #1#2{In fact, they were once diphthongs #1, and the current diphthongs were once #2}
\def \ABname{Aleksandrs Berdičevskis}
\def \AHname{Adam Hesterberg}
\def \APname{Alexander Piperski}
\def \BBname{Bozhidar Bozhanov}
\def \BIname{Boris Iomdin}
\def \DGname{Dmitry Gerasimov}
\def \HDname{Hugh Dobbs}
\def \IDname{Ivan Derzhanski}
\def \LFname{Liudmila Fedorova}
\def \MRname{Maria Rubinstein}
\def \PSname{Pavel Sofroniev}
\def \SBname{Svetlana Burlak}
\def \SGname{Stanislav Gurevich}
\def \TTname{Todor Tchervenkov}
\def \XGname{Ksenia Gilyarova}
\def \OKname{Olga Kuznetsova}
\def \edinames{\ABname, \BBname, \SBname, \IDname, \HDname, \LFname, \DGname, \XGname, \SGname, \AHname\ (\edinchef), \BIname, \APname, \MRname, \PSname, \TTname}
\def \enqueten{Name}
\def \enquetep{Place number}
\def \leafword #1{Sheet \##1}
\def \enquetea{What problems did you work on?}
\def \enqueteb{What problem did you like best?}
\def \enquetec{What problem did you find hardest?}
\def \enqueted{What problem did you find easiest?}
\def \sparecop{If you need another copy of this sheet, ask the \questor}
\def \quoted #1{“#1”}
\def \zagrad #1{(#1.)}
\def \decpoint{.}
\def \et{and}
\def \ab{or}
\def \au{or}
\def \daaracht{then}
\def \is{is}
\def \an #1{a #1}
\def \An #1{A #1}
\def \whowroti{\IDname, \HDname, \LFname, \SGname, \APname}
\def \whowrotj{\AHname}
