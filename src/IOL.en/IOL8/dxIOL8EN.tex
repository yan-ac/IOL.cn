\def \thisling{English}
\def \thislang{English}
\def \thistext{English text}
\def \olympiad{Eighth International Olympiad in Linguistics}
\def \xcountry{Sweden}
\def \yvillage{Stockholm}
\def \Julyname{July}
\def \olydates #1#2#3#4{#1–#2 #3 #4}
\def \probindl{Individual Contest Problems}
\def \solsindl{Individual Contest Solutions}
\def \probteam{Team Contest Problem}
\def \soluteam{Team Contest Solution}
\def \probword #1{Problem \##1}
\def \pontword{points}
\def \regulats{Rules for writing out the solutions}
\def \regulatx{Do not copy the statements of the problems. Write down your solution to each problem on a separate sheet or sheets. On each sheet indicate the number of the problem, the number of your seat and your surname}
\def \towarrant{Otherwise your work may be mislaid or misattributed.}
\def \regulaty{Your answers must be well-argumented. Even a perfectly correct answer will be given a low score unless accompanied by an explanation.}
\def \editorsz{Editors}
\def \edinchef{editor-in-chief}
\def \goodluck{Good luck}
\def \rulesmot{Rules}
\def \answersp{Answers}
\def \gisentra #1{The following are sentences in #1 and their English translations}
\def \andtrans{and their English translations}
\def \chaotict{in arbitrary order}
\def \fordinte #1{Translate into #1}
\def \fordinto #1{Translate into #1}
\def \corrcorr{Determine the correct correspondences}
\def \fillgaps{Fill in the gaps}
\def \filmties{Fill in the vacant cells}
\def \neshaded{you don't have to fill in the shaded ones}
\def \onBliss #1{Blissymbolics is a universal system of symbols devised by Charles K. Bliss (#1), an Australian of Austrian origin, who thought it should be understandable to all people, regardless of their native tongue}
\def \inBliss{Given are words written in Blissymbolics}
\def \xieBliss{Write in Blissymbolics}
\def \indimean{Indicate what the following symbols mean}
\def \knowbiid{knowing that two of them have the same meaning}
\def \posleft{if there is more than one symbol in the word, the mark is placed above the leftmost one}
\def \posX{part of speech}
\def \posN{noun}
\def \posA{adjective}
\def \posV{verb}
\def \composit{composition}
\def \meaning{meaning}
\def \isavowel{is a vowel}
\def \aconsons{are consonants}
\def \owumlaut{= French \word{eu} or German \word{ö}}
\def \diumlaut{= French \word{eu} and \word{u} (German \word{ö} and~\word{ü})}
\def \narschwa{$\approx$ \word{u} in \word{but}}
\def \jotsound #1{#1~= \word{y} in \word{yay!}}
\def \retrflet #1{#1 $\approx$ English \word{t} in \word{art}, uttered with the tip of the tongue turned back}
\def \velarnas #1{#1~= \word{ng} in \word{hang}}
\def \chaffric #1{#1~= \word{ch} in \word{church}}
\def \hwisound #1{#1 is a voiceless \word{w} (as \word{wh} in Scottish or Southern American \word{which})}
\def \infamily #1#2{#1 belongs to the #2 language family}
\def \lgBudukh{The Budukh language}
\def \lgDrehu{The Drehu language}
\def \toSECauc{Nakh-Daghestanian}
\def \toAustNs{Austronesian}
\def \spokenca #1#2{It is spoken by approx.\ #1 people #2}
\def \inAzerba{in Azerbaijan}
\def \inGrauCH{in the canton of Graubünden}
\def \isleLifu{on Lifu Island to the east of New Caledonia}
\def \rhaetoro{Romansh belongs to the Rhaeto-Romance subgroup of Romance}
\def \widefrit{It is one of the four national languages of Switzerland, along with German, French and Italian}
\def \inDrehu{Given are Drehu numerals in alphabetical order and their values in ascending order}
\def \writnums{Write in numerals}
\def \writDreh{Write out in Drehu}
\def \introBud{Given are verbs of the Budukh language in three forms}
\def \futtense{future tense}
\def \prohimod{prohibitive mood}
\def \nthclass #1{class #1}
\def \masculin{masculine}
\def \feminine{feminine}
\def \donVnext #1{whereby #1 depends on the vowel in the following syllable}
\def \ifendsin #1{if the stem ends in #1}
\def \wherRdef #1#2#3{where #1 is #2 if one of these consonants is found in the root, or #3 otherwise}
\def \unRfollo #1{unless #1 follows immediately}
\def \with #1{with #1}
\def \before #1{before #1}
\def \anotherC{another consonant}
\def \wiotherC #1{a cluster of #1 and another consonant}
\def \unotherC{without another consonant}
\def \postVone{after the first vowel}
\def \dialname #1{#1}
\def \Ssil{Sursilvan}
\def \Edin{Engadine}
\def \inRomansh{Given are words of two dialects of the Romansh language}
\def \somegaps{Some cells have been left blank}
\def \whatisin #1#2{What is #2 in #1}
\def \etwhatin #1{And in #1}
\def \plursare #1#2#3#4#5{In #1 #2 is #4 and #3 is #5}
\def \plurelse #1#2{You may think that it is the same in #1, but in fact the words there are #2}
\def \howexpla{How can this be explained}
\def \tranboth{Translate into both dialects}
\def \bothdial{in both dialects}
\def \brokrule #1#2{In #1 (unlike #2) the first rule doesn't apply in plural forms}
\def \hypfirst{This may mean that it doesn't work if one consonant is part of the stem and the other belongs to the ending}
\def \hyptwond{or that the vowel is chosen before the ending is added}
\def \hypthird{or that the vowel in the plural is made to match the vowel in the singular}
\def \inGenCod{One of the major achievements in genetics was the decipherment of the genetic code—the creation of an mRNA–polypeptide dictionary}
\def \proteins{Polypeptides (proteins) are building blocks of all living organisms}
\def \aminacid #1{Polypeptide molecules are chains that consist of amino acids (denoted as #1 etc.), and it is the sequence of amino acids in the polypeptide that determines its properties}
\def \nucletid #1{When cells synthesize polypeptides, they follow instructions written in molecules of messenger ribonucleic acid (mRNA), chains that consist of four nucleotides (denoted as #1)}
\def \ifuseRNA{If a cell uses as a template the following mRNA sequence}
\def \synthese{the following polypeptides will be synthesized}
\def \anuseRNA{A cell uses the following mRNA sequence}
\def \syntwhat{What polypeptide(s) will it synthesize}
\def \synthest{A cell synthesized the following polypeptide}
\def \usedwhat{What mRNA sequence(s) could it have used}
\def \rootypes{The nucleotide pairs are sometimes called \textbf{roots} and classified into two groups}
\def \sicroots #1{#1 roots}
\def \forPlNom{strong}
\def \debPlNom{weak}
\def \egroots #1#2{Examples of #1 roots are #2}
\def \forPlGen{strong}
\def \debPlGen{weak}
\def \definfor #1#2{A root #1 is strong if #2 encode the same amino acid}
\def \defindeb{A root is weak if this is not the case}
\def \claroots{Classify all the other roots}
\def \mRNAwith #1{All mRNA sequences start with #1}
\def \polycont #1{The four polypeptides in the example consist of #1 amino acids}
\def \fragcont #1{and the mRNA sequence contains #1 nucleotides}
\def \trioprob{It appears probable that three nucleotides (a triplet) denote one amino acid or are a separator between polypeptides (in reality a signal to terminate synthesis)}
\def \butalloc #1#2{However, since there are #1 possible triplets (all but two of which are present in the example) and only #2 different amino acids, some triplets have the same meaning}
\def \waitillc{The sequence contains both nucleotide triplets that were absent from the example, so we cannot be sure in the answer, but it will be confirmed when we have solved the problem to the end}
\def \simpdata{The data presented here are slightly simplified}
\def \possties{possibilities}
\def \pointuse #1{Pointers #1 are used to refer to specific parts of the symbols}
\def \pointer{pointer}
\def \pointers{pointers}
\def \surwards{upwards}
\def \subwards{downwards}
\def \outwards{outwards}
\def \inMongol #1{Consider the following words and their explications taken from a monolingual Mongolian dictionary (#1), given in Roman transliteration}
\def \tranmost{Translate as many Mongolian words from the text as you can}
\def \timeword{time}
\def \stomach{stomach}
\def \woman{woman}
\def \world{world}
\def \nummer{number}
\def \montpass{mountain pass}
\def \filling{tooth filling}
\def \mashnoun{mash, pulp}
\def \puolpa{dried meat}
\def \fuorma{form}
\def \labour{labour}
\def \shortmot{short}
\def \coulant{generous}
\def \showverb{to show}
\def \totall{all}
\def \finally{finally}
\def \discuors{conversation}
\def \elmsing{elm}
\def \namesing{name}
\def \ugolsing{angle}
\def \florsing{flower}
\def \parents{parents}
\def \elmplur{elms}
\def \nameplur{names}
\def \ugolplur{angles}
\def \florplur{flowers}
\def \overtake{overtake}
\def \alqol{sit down}
\def \arxar{sleep}
\def \qalqal{lie, recline}
\def \sonkon{be startled}
\def \ynkan{remain}
\def \jechi{cross, go across}
\def \woltu{tie}
\def \halghu{swallow}
\def \harki{set on (animals)}
\def \horchu{push}
\def \jolku{make to roll}
\def \quroghu{bring to a halt}
\def \osu{put}
\def \chighi{carry, lead}
\def \chorhucu{exchange}
\def \ensi{extinguish}
\def \liquid{liquid}
\def \water{water}
\def \airstuff{air}
\def \gasstuff{gas}
\def \fogstuff{fog}
\def \bigcircS{circle}
\def \aurinkoS{sun}
\def \headneck{head with neck}
\def \eyezbrow{eye with eyebrow}
\def \bodytors{body (torso)}
\def \waist{waist}
\def \heart{heart}
\def \necknoun{neck}
\def \eyebrow{eyebrow}
\def \eyenoun{eye}
\def \nosenoun{nose}
\def \mouth{mouth}
\def \lipsnoun{lips}
\def \saliva{saliva}
\def \eastside{east}
\def \western{western}
\def \merry{merry}
\def \malmerry{sad}
\def \sickill{ill, sick}
\def \blowverb{to blow}
\def \breathe{to breathe}
\def \riseverb{to rise}
\def \cryweep{to weep}
\def \activity{activity}
\def \activadj{active}
\def \activerb{to act, be active}
\def \ABname{Alexander Berdichevsky}
\def \AHname{Adam Hesterberg}
\def \ANname{Aleksei Nazarov}
\def \APname{Alexander Piperski}
\def \BBname{Bozhidar Bozhanov}
\def \BIname{Boris Iomdin}
\def \DGname{Dmitry Gerasimov}
\def \IDname{Ivan Derzhanski}
\def \LFname{Ludmilla Fedorova}
\def \MRname{Maria Rubinstein}
\def \RPname{Renate Pajusalu}
\def \SBname{Svetlana Burlak}
\def \SGname{Stanislav Gurevich}
\def \TTname{Todor Tchervenkov}
\def \XGname{Ksenia Gilyarova}
\def \edinames{\ABname, \BBname, \SBname, \IDname, \LFname, \DGname, \XGname, \SGname, \AHname, \BIname, \ANname, \RPname, \APname\ (\edinchef), \MRname, \TTname}
\def \quoted #1{“#1”}
\def \enqueten{Name}
\def \enquetep{Place}
\def \enquetea{What problems did you work on?}
\def \enqueteb{What problem did you like best?}
\def \enquetec{What problem did you find hardest?}
\def \enqueted{What problem did you find easiest?}
\def \et{and}
\def \ab{or}
\def \au{or}
\def \whowroti{\ABname, \IDname, \XGname, \BIname, \APname}
\def \whowrotj{\BIname}
