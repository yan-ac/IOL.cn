\def \thistext{中文文本}
\def \nthIOL #1{#1国际语言学奥林匹克竞赛}
\def \thisth{第十届}
\def \thisland{斯洛文尼亚}
\def \thistown{卢布尔雅那}
\def \Julyname{7月}
\def \Auguname{8月}
\def \olydates #1#2#3#4#5{#5年#2#1日 — #4#3日}
\def \leafword #1{答题纸 \##1}
\def \probword #1{题 \##1}
\def \probsing #1{#1 题目}
\def \probplur #1{#1 题目}
\def \respsing #1{#1 答案}
\def \solusing #1{#1 解答}
\def \soluplur #1{#1 解答}
\def \indicont{个人赛}
\def \teamcont{团体赛}
\def \Teamword{团队}
\def \pontword{分}
\def \regulats{解题规则}
\def \regulado{毋需抄题}
\def \regulare{将不同问题的解答分述于不同的答题纸上}
\def \regulami{每张纸上注明题号、座位号和姓名}
\def \towarrant{否则答题纸可能被误放或遗失}
\def \regulaty{解答需详细论证}
\def \regulatz{无解释之答案, 即便完全正确, 也会被处以低分}
\def \editorsz{编者}
\def \edinchef{主编}
\def \rulesmot{规则}
\def \answersp{答案}
\def \enquetex{问卷}
\def \enqueten{姓名}
\def \enquetep{座位号}
\def \enquetea{你做了哪些题目}
\def \enqueteb{你最喜欢哪道题}
\def \enquetec #1{你觉得哪道题#1}
\def \seemhard{最难}
\def \seemeasy{最简单}
\def \goodluck{祝你好运}
\def \quoted #1{“#1”}
\def \giveland #1{下列是#1个老挝语的国名}
\def \findland{辨认这些国家}
\def \guessond{猜测老挝语中这些国名的发音}
\def \givesent #1{下列是#1的语句}
\def \inBasque{巴斯克语}
\def \indiCent #1{#1的中部方言}
\def \giwordss #1{下列是#1的词和词组}
\def \ofRotuma{罗图马语}
\def \andtrans #1{及其#1翻译}
\def \chaotict{(乱序排列)}
\def \gyRotuma{以下是罗图马语的七个身体部位的名字}
\def \pasoreus{其中一句汉语对应两个巴斯克语语句}
\def \formahat #1{其中一句汉语有多种#1翻译}
\def \findtran{找出该语句并给出其它可能的翻译}
\def \thremore #1{以下是另外三个#1的词}
\def \fordinto #1{翻译成#1}
\def \fordouta #1{将括号外的语句翻译成#1}
\def \tothislang{汉语}
\def \tolgEus{巴斯克语}
\def \tolgRtm{罗图马语}
\def \TheRtm{罗图马语}
\def \Thelg #1{#1}
\def \tolg #1{#1}
\def \lgUndu{温布-尤恩固语}
\def \lgTeop{蒂奥普语}
\def \Dyirbal{迪尔巴尔语}
\def \infamily #1#2{#1属于#2}
\def \toAustNs{南岛语系}
\def \toTNGfam{泛新几内亚语系}
\def \toPNyfam{帕马-恩永甘语系}
\def \spokenca #1#2{#2, 大约#1人使用该语言}
\def \inPNG{在巴布亚新几内亚}
\def \inFiji{在斐济}
\def \spokNEQL{它是一种已经消亡的澳大利亚昆士兰州东北部土著语言}
\def \corrcorr{找出正确的对应关系}
\def \writnums{用阿拉伯数字表示}
\def \writeout #1{翻译成#1}
\def \canttran #1#2{仅仅根据上述材料, 你不能肯定#2的#1翻译}
\def \whatposs #1{这些词的#1翻译理论上有哪些}
\def \stateopa #1{一位语言学家打算写出#1的语法规则}
\def \stateope{首先, 她让信息人将这些独立的语句翻译成其母语}
\def \stateopi{以下是她所得到的}
\def \stateopo #1{随后, 这位语言学家记录了#1的自发性言语并加入了一些新的语法规则}
\def \stateopu #1{以下是这些#1对话的摘录}
\def \stateopy{其情境在括号中给出}
\def \bonaiana{这条鱼}
\def \bonaekae{这个包}
\def \bonaykae{\bonaekae}
\def \bonaover{这个椰子}
\def \paani #1#2{\iN{#1}吃了#2}
\def \teopdatg{你们听到了他}
\def \epaatara #1{看见了#1}
\def \teopdatu #1{看见了#1}
\def \teopdatf #1{我看见了#1}
\def \teopdatt{他看见了你们}
\def \who{主语}
\def \whotom{谁}
\def \what{什么}
\def \hee{给}
\def \enaaphee #1#2{我把#2给了#1}
\def \oteitomu{男人}
\def \beikoemu{孩子}
\def \heesoasi #1{把#1给了这个男孩}
\def \teopdati #1{把#1给了你们}
\def \teopdatm #1{我们把#1给了你}
\def \oraoraat{这个男巫}
\def \whom{谁}
\def \withwhat{用什么}
\def \tasu{打}
\def \teopdatc{我们打了那个孩子}
\def \teopdata{你打了我}
\def \teopdatj{我用石头打了你}
\def \teopdats #1{你用包打了#1}
\def \teopdatw{用椰子打了这个男孩}
\def \asun{杀}
\def \teopdato{我用它杀了他}
\def \litt{字面意义: }
\def \withim{用他}
\def \toraaram{用这个斧头}
\def \teopdatk #1{他们#1杀了这个女人}
\def \paaqasun #1#2{#2杀了#1}
\def \teopdate{杀了他}
\def \dao{称呼}
\def \whomcall{谁}
\def \whatcall{什么}
\def \teopdatl{我们称这个男孩为男巫}
\def \teopdatn{他称我为小孩}
\def \teopdaty{他们称这位男巫女人}
\def \thecerer{这位男巫}
\def \theboy{这个男孩}
\def \Theman{这个男人}
\def \men{男人们}
\def \Thewoman{这个女人}
\def \women{女人们}
\def \alltrest{所有其它东西}
\def \Why{为什么}
\def \didsocry #1{#1哭得这么厉害}
\def \offended #1{#1被冒犯了}
\def \teopdaqz{这个斧头是湿的}
\def \teopdaqv{这里没有留下任何食物}
\def \tabaqani{这些食物}
\def \tabaqany{\tabaqani}
\def \teopdaqu{接下来这个女人发生了什么}
\def \teopdaqx{这个包在哪里}
\def \mintzatu{与……谈了}
\def \mintzat{宾语}
\def \mintzaty #1#2{\iN{#1}与\iA{#2}谈了}
\def \hurbildu{与……接洽了}
\def \hurbild{宾语}
\def \hurbildy #1#2{\iN{#1}与\iA{#2}接洽了}
\def \ahaztu{忘了}
\def \ahazt{宾语}
\def \ahazty #1#2{\iN{#1}忘了\iA{#2}}
\def \lagundu{帮助了}
\def \lagund{宾语}
\def \lagundy #1#2{\iN{#1}帮助了\iA{#2}}
\def \ukitu{触摸了}
\def \ukit{宾语}
\def \ukity #1#2{\iN{#1}触摸了\iA{#2}}
\def \iN #1{\6#1(我,你,他,我们,你们,他们)}
\def \iA #1{\6#1(我,你,他,我们,你们,他们)}
\def \iD #1{#1}
\def \iJ #1{#1}
\def \dyirbalc{糖正在使这位健康的继母发胖}
\def \dyirbalh{酒正在使这个疲倦的医生入睡}
\def \dyirbale{这缕烟正在使这只蝎子感到疲倦}
\def \dyirbalf{这位被冒犯的父亲正在搬运这瓶酒}
\def \dyirbali{这只蜥蜴正在跟随这缕烟}
\def \dyirball{这个肥胖的医生正在推这块巨石}
\def \dyirbalj{这只蜻蜓正在搜寻这棵刺痛树}
\def \dyirbald{这个疲倦的沙袋鼠正在搜寻这片小羽毛}
\def \dyirbaln{这只巨大的蚱蜢正在搜寻这个弯掉的矛}
\def \dyirbalo{这个安静的男孩正在看着这只疲倦的蜥蜴}
\def \dyirbalq{这只小沙袋鼠正在看着这只蜻蜓}
\def \dyirbals{这只睡着的袋貂正在无视着这嘈杂的噪音}
\def \dyirbalp{这个总是被跟随的男人正在冒犯这位强壮的父亲}
\def \dyirbalr{这位一直被跟随的阿姨正在弯曲这片羽毛}
\def \dyirbalm{这位总是在推石头的继母正在看着这缕烟}
\def \dyirbalt{这只毛毛虫正在搜寻这个一直在搬石头的男人}
\def \dyirbalv{这只蜥蜴正在搜寻这块一直被推的石头}
\def \dyirbalu{这位总是被搜寻的父亲正在治疗这个女孩}
\def \dyirbala{酒正在使这个一直被责骂的男人感到疲倦}
\def \dyirbalw{这个总是责备医生的男人正在跟随这只沙袋鼠}
\def \dyirbalg{这条总是在搜寻蝎子的死蛇正在跟随这只袋貂}
\def \dyirbalb{这个强壮的男人正在责骂这位总是跟随死蛇的母亲}
\def \dyirbalk{这位总是被无视的母亲正在搬运这块糖}
\def \ngaymunga #1#2{总是在#1 #2}
\def \munga #1{总是被#1}
\def \bayimbam{毛虫}
\def \bayimbap{幼虫}
\def \auntword{阿姨}
\def \mugunanj{母亲的姐姐}
\def \muqgamot{噪音}
\def \defdeadd{死蛇是一种澳大利亚的毒蛇}
\def \defwalla{沙袋鼠是一种小动物, 袋鼠的近亲}
\def \defpossu{袋貂是一种澳大利亚树栖的有袋目动物}
\def \defstree{刺痛树是一属有带刺茸毛的灌木和树, 其中一些对人类有害}
\def \thoughtE #1{一位语言学家认为上述#1语句中存在一处错误}
\def \nonerror{实际上, 这里并没有错误}
\def \wheremis{你认为这位语言学家觉得哪里错了}
\def \theremis #1#2{这位语言学家认为样例#2中的#1是一处错误}
\def \babamyth #1#2{之所以这位语言学家感到怪异, 是因为一种动物在#1的神话中被当作#2}
\def \oldwomen{老女人}
\def \oldwoman{老女人}
\def \bundinye{蚱蜢既不是女人, 也不是一种危险动物, 但使用了与两者相同的冠词}
\def \mustmyth #1{所以它一定是神话里的#1}
\def \animals{动物}
\def \dangerux #1{危险的#1和物体}
\def \whatanim{这是哪种动物}
\def \defcopra{干椰肉是干的椰子核}
\def \niu{干椰肉}
\def \niut #1{#1干椰肉}
\def \qolo{切}
\def \ququ{手臂/手}
\def \fau{年}
\def \aqoffau{年末}
\def \faega{词}
\def \huga{心}
\def \kia{脖子}
\def \hunkia{脖子底部}
\def \huni{低端}
\def \leva{头发}
\def \leavpiri{卷发}
\def \piri{卷曲(动词)}
\def \susu{胸}
\def \issusu{乳头}
\def \isu{鼻子}
\def \mafa{眼睛}
\def \maftiro{眼镜}
\def \tiro{玻璃}
\def \matmamas{冰}
\def \matiti{寒冷(名词)}
\def \mofa{垃圾}
\def \moafmofa{丢垃圾}
\def \poga{洞}
\def \poagpoga{布满洞的}
\def \pala{穿孔}
\def \nusutiro{窗户}
\def \nusura{门}
\def \pogi{夜晚}
\def \pulu{胶水}
\def \hafpraki{火山岩}
\def \hafu{岩石}
\def \hafhafu{布满岩石的}
\def \ruhuga{胃痛}
\def \toqa{英雄}
\def \kalkalu{圆的}
\def \kaluV{包围(动词)}
\def \kaluN{手镯}
\def \lala{深的}
\def \qelqele{浅的}
\def \huagqele{没有耐心的}
\def \huaglala{耐心的}
\def \huagtoqa{有勇气的}
\def \isqa{尖顶的}
\def \mafpogi{盲的}
\def \mamasa{固体}
\def \pulpulu{黏糊糊的}
\def \riamrima{锃亮的}
\def \qele{靠近(动词)}
\def \aqofi{使精疲力竭}
\def \faegay{说话(动词)}
\def \faeagqu{使用手语}
\def \huli{翻转(动词)}
\def \puhraki{使沸腾}
\def \rima{闪光}
\def \hulhafu{吹}
\def \hurrican{飓风}
\def \adjeverb #1{使#1}
\def \persnumb #1#2{第\ifcase #1\or 一\or 二\or 三\fi 人称#2}
\def \Sg{单数}
\def \Pl{复数}
\def \Perspron{人称代词}
\def \wordsord{语序是: }
\def \sestruct{这些语句的结构为}
\def \nounword{名词}
\def \adjectif{形容词}
\def \Verb{动词}
\def \Pred{predicate}
\def \Sb{主语}
\def \Ob{宾语}
\def \bis #1{另一个#1}
\def \attribut{修饰词}
\def \attribum{被修饰}
\def \ergatend{主语及其定语有后缀}
\def \worderNA{形容词跟在名词后面}
\def \incompoN{In a noun-noun compound}
\def \modihead{the second part modifies the first one}
\def \artnomen{每个名词前都有一个冠词}
\def \evelbona #1#2{如果是语句中第一个第三人称, 则为#1, 否则为#2}
\def \selepron #1#2{#1的代词的冠词, #2, 也是以同样的方式决定的}
\def \topicfrt #1{在上文对话中提到的词将被移到句首, 并使用冠词#1}
\def \pronoffe #1#2{如果因此, 一个代词#1被移至动词的后面, 它将会丢掉开头的#2}
\def \nounoffe #1#2{如果移动的#1是一个名词, 它将保留它的冠词#2}
\def \temathes{当两个词构成一个短语时, 第一个词的形态将发生如下变化}
\def \adjedoub{同样的情况发生在通过重复一个名词或动词构成一个形容词时}
\def \alsomean #1{也有#1的意思}
\def \hewhohas #1#2{拥有#1 #2的人}
\def \nounverb{产生的复合词可以为动词或者名词}
\def \simotend #1{如果这个词以#1结尾}
\def \byavowel{\vocal}
\def \byacsant{\const}
\def \vocal{元音}
\def \const{辅音}
\def \contains #1{有#1}
\def \twosylla{两个音节}
\def \trosylla{超过两个音节}
\def \samplart #1#2{是发音部位与#2相同的浊#1}
\def \stopcons{塞音}
\def \nasalcns{鼻音}
\def \wordlast{该词结尾最后一个音}
\def \velarnas{= 普通话 \word{杭} (\word{háng})中的\word{ng}}
\def \palatnas{= 普通话 \word{娘} (\word{niáng})中的\word{ni}}
\def \glotstop #1{#1是一个辅音(声门塞音)}
\def \oshiroko{= 普通话 \word{欧}}
\def \aeligatu #1{#1 $\approx$~\word{crack}里的\word a}
\def \oeumlaut{= 普通话 \word{约} 的元音}
\def \ueumlaut{= 普通话 \word{于} 的元音}
\def \longmark #1{标记~#1表示长元音}
\def \et #1{和#1}
\def \ett #1{且#1}
\def \ab{或}
\def \au{或}
\def \like{如}
\def \APname{Alexander Piperski}
\def \BIname{Boris Iomdin}
\def \DGname{Dmitry Gerasimov}
\def \IDname{戴谊凡}
\def \PSname{Pavel Sofroniev}
\def \XGname{Ksenia Gilyarova}
\def \LFname{Liudmila Fedorova}
\def \MKname{Maria Konoshenko}
\def \NZname{Natalya Zaika}
\def \SBname{Svetlana Burlak}
\def \MRname{Maria Rubinstein}
\def \ABname{Aleksandrs Berdičevskis}
\def \LPname{Aleksejs Peguševs}
\def \ASname{Artūrs Semeņuks}
\def \BLname{Bruno L’Astorina}
\def \HDname{Hugh Dobbs}
\def \GHname{Gabrijela Hladnik}
\def \RSname{Rosina Savisaar}
\def \JLname{李在揆}
\def \LMSname{刘闽晟}
\def \CQTname{曹起曈}
\def \edinames{\ABname, \SBname, \IDname(\edinchef), \HDname, \LFname, \DGname, \XGname, \GHname, \BIname, \BLname, \JLname, \LPname, \APname, \MRname, \RSname, \ASname, \PSname}
\def \whowroti{\LMSname, \CQTname, \IDname}
\def \idLand{印度尼西亚}
\def \kpLand{朝鲜}
\def \ngLand{尼日利亚}
\def \afLand{阿富汗}
\def \thLand{泰国}
\def \fiLand{芬兰}
\def \skLand{斯洛伐克}
\def \lvLand{拉脱维亚}
\def \cnLand{中国}
\def \ghLand{加纳}
\def \iqLand{伊拉克}
\def \ieLand{爱尔兰}
\def \yeLand{也门}
\def \pkLand{巴基斯坦}
\def \loLand{老挝}
\def \dzLand{阿尔及利亚}
\def \isLand{冰岛}
\def \npLand{尼泊尔}
\def \dkLand{丹麦}
\def \snLand{塞内加尔}
\def \inLand{印度}
\def \azLand{阿塞拜疆}
\def \noLand{挪威}
\def \gtLand{危地马拉}
\def \cmLand{喀麦隆}
\def \aeLand{阿联酋}
\def \ilLand{以色列}
\def \coLand{哥伦比亚}
\def \soLand{索马里}
\def \krLand{韩国}
\def \nlhLand{荷兰}
\def \veLand{委内瑞拉}
\def \srLand{塞尔维亚}
\def \zaLand{南非}
\def \alLand{阿尔巴尼亚}
\def \cuLand{古巴}
\def \peLand{迷路}
\def \joLand{约旦}
\def \luLand{卢森堡}
\def \vnLand{越南}
\def \saLand{沙特阿拉伯}
\def \nzLand{新西兰}
\def \itLand{意大利}
\def \amLand{亚美尼亚}
\def \syLand{叙利亚}
\def \irLand{伊朗}
\def \bgLand{保加利亚}
\def \uzLand{乌兹别克斯坦}
\def \geLand{格鲁吉亚}
\def \trLand{土耳其}
\def \maLand{摩洛哥}
\def \caLand{加拿大}
\def \usLand{美国}
\def \keLand{肯尼亚}
\def \ptLand{葡萄牙}
\def \boLand{玻利维亚}
\def \mdLand{摩尔多瓦}
\def \nlLand{尼德兰}
